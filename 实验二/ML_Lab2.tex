\documentclass[withoutpreface,bwprint]{cumcmthesis}



\usepackage{float}
\usepackage{url}
\usepackage{booktabs}  
\usepackage{threeparttable} 
\usepackage{parskip}
\usepackage{subfigure}
\usepackage{epstopdf}
\usepackage{indentfirst}
\usepackage{graphics}
\setlength{\parindent}{1.5em}
\begin{document}
\section{实验目的}
理解逻辑回归模型,掌握逻辑回归的参数估计算法。测试一下满足和不满足朴素贝叶斯的情况下逻辑回归的效果,并找一组UCI上的数据集观测效果。

\section{实验要求及实验环境}
\subsection{实验要求}
\begin{enumerate}
\setlength{\itemindent}{1.5em}
\item 使用梯度下降,牛顿法,共轭梯度法一种或多种方法求解;
\item 手动生成符合多元高斯分布的数据,要求有满足和不满足朴素贝叶斯两种;
\item 在UCI网站上找一组数据集观测效果;
\item 使用正则项防止过拟合;
\end{enumerate}

\subsection{实验环境}
\begin{itemize}
\setlength{\itemindent}{1.5em}
\item 硬件:Intel i5-8265U、512G SSD、8G RAM;
\item 系统:Windows 11;
\item IDE:Pycharm。
\end{itemize}

\section{设计思想}
\subsection{算法原理}
对某一个样本$<\boldsymbol{x},y>$,其标记为1的概率为
\begin{equation*}
\begin{aligned}
P(y=1|\boldsymbol{x})&=\frac{P(y=1)P(\boldsymbol{x}|y=1)}{P(y=1)P(\boldsymbol{x}|y=1)+P(y=0)P(\boldsymbol{x}|y=0)}\\
&=\cfrac{1}{1+\cfrac{P(y=0)P(\boldsymbol{x}|y=0)}{P(y=1)P(\boldsymbol{x}|y=1)}} \\
&=\cfrac{1}{1+\exp(\ln\cfrac{P(y=0)}{P(y=1)}+\ln\cfrac{P(\boldsymbol{x}|y=0)}{P(\boldsymbol{x}|y=1)})}.
\end{aligned}
\end{equation*}

令$ P(y=1)=\pi $,假设此时样本各维度间互相独立。则有
\begin{equation}
P(y=1|\boldsymbol{x})=\cfrac{1}{1+\exp(\ln\cfrac{1-\pi}{\pi}+\sum_{i=1}^{m}\ln\cfrac{P(x_{i}|y=0)}{P(x_{i}|y=1)})},
\end{equation}
其中$m$为样本$\boldsymbol{x}$的属性个数。

不妨设$P(x_{i}|y=1)\sim N(\mu_{i1},\sigma_{i})$,$P(x_{i}|y=0)\sim N(\mu_{i0},\sigma_{i})$,则:
\begin{equation*}
P(x_{i}|y=1)=\frac{1}{\sqrt{2\pi}\sigma_{i}}e^{-\frac{(x_{i}-\mu_{i1})^{2}}{2\mu_{i}^{2}}},
\end{equation*}
\begin{equation*}
P(x_{i}|y=0)=\frac{1}{\sqrt{2\pi}\sigma_{i}}e^{-\frac{(x_{i}-\mu_{i0})^{2}}{2\mu_{i}^{2}}},
\end{equation*}
\begin{equation*}
\ln\cfrac{P(x_{i}|y=0)}{P(x_{i}|y=1)})=-\frac{(x_{i}-\mu_{i0})^{2}}{2\mu_{i}^{2}}+\frac{(x_{i}-\mu_{i1})^{2}}{2\mu_{i}^{2}}=\frac{2(\mu_{i0}-\mu_{i1})x_{i}+\mu_{i1}^{2}-\mu_{i0}^{2}}{2\sigma_{i}^{2}}.
\end{equation*}

对式(1)进一步化简可得
\begin{equation*}
\begin{aligned}
P(y=1|\boldsymbol{x})&=\cfrac{1}{1+\exp(\ln\cfrac{1-\pi}{\pi}+\sum_{i=1}^{m}
(\frac{2(\mu_{i0}-\mu_{i1})}{2\sigma_{i}^{2}}x_{i}+\frac{\mu_{i1}^{2}-\mu_{i0}^{2}}{2\sigma_{i}^{2}}))}\\
&=\frac{1}{1+exp(w_{0}+\sum_{i=1}^{m}w_{i}x_{i})}.
\end{aligned}
\end{equation*}

构造
\begin{equation*}
\boldsymbol{w}'=\begin{pmatrix}
w_{0} & w_{1} &\cdots & w_{m}\\
\end{pmatrix},
\end{equation*}
\begin{equation*}
\boldsymbol{x}'=\begin{pmatrix}
1 & x_{1} &\cdots & x_{m}\\
\end{pmatrix},
\end{equation*}

则式(1)可进一步简化为
\begin{equation*}
P(y=1|\boldsymbol{x})=\frac{1}{1+\exp(\boldsymbol{w}'\boldsymbol{x})}.
\end{equation*}

由上易得
\begin{equation*}
P(y=0|\boldsymbol{x})=\frac{\exp(\boldsymbol{w}'\boldsymbol{x})}{1+\exp(\boldsymbol{w}'\boldsymbol{x})},
\end{equation*}
\begin{equation*}
\frac{P(y=0|\boldsymbol{x})}{P(y=1|\boldsymbol{x})}=\exp(\boldsymbol{w}'\boldsymbol{x}),
\end{equation*}
\begin{equation*}
\ln\frac{P(y=0|\boldsymbol{x})}{P(y=1|\boldsymbol{x})}=\boldsymbol{w}'\boldsymbol{x}.
\end{equation*}

现在有一组样本点$\{<\boldsymbol{x_{1}},y_{1}>,<\boldsymbol{x_{2}},y_{2}> \cdots<\boldsymbol{x_{n}},y_{n}>\}$,应找到一组$\boldsymbol{w}$使得上面这一组数据出现的概率最大,即
\begin{equation}
\begin{aligned}
W_{MLE}&=\arg\max P(<\boldsymbol{x_{1}},y>,<\boldsymbol{x_{2}},y> \cdots<\boldsymbol{x_{n}},y>|\boldsymbol{w})\\
&=\arg\max\prod_{i=1}^{n}P(<\boldsymbol{x_{i}},y_{i}>|\boldsymbol{w})
\end{aligned}
\end{equation}

但之前仅推出$P(y_{i}|\boldsymbol{x_{i}})$,因此将式(2)变化为
\begin{equation*}
W_{MCLE}=\arg\max\prod_{i=1}^{n}P(y_{i}|\boldsymbol{x_{i}},\boldsymbol{w}).
\end{equation*}

而加入先验概率,采用贝叶斯估计之后
\begin{equation*}
W_{MCAP}=\arg\max\prod_{i=1}^{n}P(\boldsymbol{w})P(y_{i}|\boldsymbol{x_{i}},\boldsymbol{w}).
\end{equation*}

具体表现为损失函数则为
\begin{equation*}
\begin{aligned}
L(\boldsymbol{w})_{MCLE}
&=\sum_{i=1}^{n}y_{i}\ln P(y_{i}=1|\boldsymbol{x_{i}})+(1-y_{i})\ln P(y_{i}=0|\boldsymbol{x_{i}})\\
&=\sum_{i=1}^{n}(1-y_{i})\boldsymbol{w'x_{i}}-\ln[1+\exp(\boldsymbol{w'x_{i}})].
\end{aligned}
\end{equation*}

对其求导有
\begin{equation}
\begin{aligned}
\frac{\partial{L(\boldsymbol{w})}}{\partial{\boldsymbol{w}}}&=\sum_{i=1}^{n}[\frac{1}{1+\exp(\boldsymbol{w'x_{i}})}-y_{i}]\boldsymbol{x_{i}}\\
&=\sum_{i=1}^{n}[sigmoid(-\boldsymbol{w'x_{i}})-y_{i}]\boldsymbol{x_{i}}.
\end{aligned}
\end{equation}

为后续编程方便,现将其改写为矩阵乘法的方式。首先构造
\begin{equation*}
\boldsymbol{X}=
\begin{bmatrix}
x_{11}  &x_{12}  & \cdots & x_{1(m+1)}\\ 
x_{21}  &x_{22}  & \cdots & x_{2(m+1)}\\ 
\vdots  &\vdots  &        &\vdots \\ 
x_{n1}  &x_{n2}  & \cdots & x_{n(m+1)}\\ 
\end{bmatrix}
\in \boldsymbol{R}^{n*\left(m+1\right)},
\end{equation*}

\begin{equation*}
{\boldsymbol{y}}'=
\begin{pmatrix}
y_{1} & y_{2} &\cdots & y_{n}\\
\end{pmatrix},
\end{equation*}
$\boldsymbol{X}$,$\boldsymbol{y}$中每一行表示一个样本,$x_{i0}=1$,一个样本有$m$个属性,一个标记,一共有$n$个样本。

则式(3)可以改写为
\begin{equation*}
\frac{\partial{L(\boldsymbol{w})}}{\partial{\boldsymbol{w}}}
=k_{1}\boldsymbol{x_{1}}+k_{2}\boldsymbol{x_{2}}+\cdots+k_{n}\boldsymbol{x_{n}}=\boldsymbol{X'K},
\end{equation*}

\begin{equation*}
{\boldsymbol{X'}}=
\begin{pmatrix}
\boldsymbol{x_{1}} & \boldsymbol{x_{2}} &\cdots & \boldsymbol{x_{n}}\\
\end{pmatrix},
\end{equation*}

\begin{equation*}
{\boldsymbol{K}}=
\begin{pmatrix}
k_{1}\\k_{2}\\ \vdots\\ k_{n}
\end{pmatrix},
\end{equation*}

\begin{equation*}
k_{i}=\frac{1}{1+\exp(\boldsymbol{w'x_{i}})}-y_{i}=sigmoid(-\boldsymbol{w'x_{i}})-y_{i}.
\end{equation*}

即
\begin{equation*}
\frac{\partial{L(\boldsymbol{w})}}{\partial{\boldsymbol{w}}}=\boldsymbol{X'}[sigmoid(-\boldsymbol{Xw})-\boldsymbol{Y}].
\end{equation*}

其二阶导数为
\begin{equation*}
\frac{\partial^{2}{L(\boldsymbol{w})}}{\partial{\boldsymbol{w}}\partial{\boldsymbol{w'}}}=\boldsymbol{X'}\frac{\partial{sigmoid(-\boldsymbol{Xw})}}{\partial{\boldsymbol{w'}}}.
\end{equation*}

其中
\begin{equation*}
sigmoid(-\boldsymbol{Xw})=\begin{pmatrix}
sigmoid(-\boldsymbol{x_{1}'w})\\sigmoid(-\boldsymbol{x_{2}'w})\\ \vdots\\ sigmoid(-\boldsymbol{x_{n}'w})
\end{pmatrix}
=\begin{pmatrix}
z_{1}\\ z_{2} \\ \vdots \\ z_{n}
\end{pmatrix}
\end{equation*}


\begin{equation*}
\boldsymbol{w'}=\begin{pmatrix}
w_{1}&w_{2}&\cdots&w_{m+1}\end{pmatrix}
\end{equation*}

列向量对行向量求导有
\begin{equation}
\frac{\partial{sigmoid(-\boldsymbol{Xw})}}{\partial{\boldsymbol{w'}}}=
\begin{bmatrix}
\frac{\partial{z_{1}}}{\partial{w_{1}}}&\frac{\partial{z_{1}}}{\partial{w_{2}}}&\cdots&\frac{\partial{z_{1}}}{\partial{w_{m+1}}}\\
\frac{\partial{z_{2}}}{\partial{w_{1}}}&\frac{\partial{z_{2}}}{\partial{w_{2}}}&\cdots&\frac{\partial{z_{2}}}{\partial{w_{m+1}}}\\
\vdots&\vdots& &\vdots\\
\frac{\partial{z_{n}}}{\partial{w_{1}}}&\frac{\partial{z_{n}}}{\partial{w_{2}}}&\cdots&\frac{\partial{z_{n}}}{\partial{w_{m+1}}}
\end{bmatrix}
\end{equation}

而
\begin{equation*}
\frac{\partial{z_{i}}}{\partial{w_{j}}}=\frac{\partial{sigmoid(-\boldsymbol{x_{1}'w})}}{\partial{w_{j}}}=z_{i}(1-z_{i})x_{ij}
\end{equation*}

则式(4)求解得
\begin{equation*}
\frac{\partial{sigmoid(-\boldsymbol{Xw})}}{\partial{\boldsymbol{w'}}}=
\begin{bmatrix}
z_{1}(1-z_{1})x_{11}&z_{1}(1-z_{1})x_{12}&\cdots&z_{1}(1-z_{1})x_{1(m+1)}\\
z_{2}(1-z_{2})x_{21}&z_{2}(1-z_{2})x_{22}&\cdots&z_{2}(1-z_{2})x_{2(m+1)}\\
\vdots&\vdots& &\vdots\\
z_{n}(1-z_{n})x_{n1}&z_{n}(1-z_{n})x_{n2}&\cdots&z_{n}(1-z_{n})x_{n(m+1)}\\
\end{bmatrix}
\end{equation*}

构造
\begin{equation*}
\boldsymbol{Z}=\begin{pmatrix}
sigmoid(-\boldsymbol{x_{1}'w})(1-sigmoid(-\boldsymbol{x_{1}'w}))\\sigmoid(-\boldsymbol{x_{2}'w})(1-sigmoid(-\boldsymbol{x_{2}'w}))\\ \vdots \\sigmoid(-\boldsymbol{x_{n}'w})(1-sigmoid(-\boldsymbol{x_{n}'w}))
\end{pmatrix}
\end{equation*}

则二阶导数可写作
\begin{equation*}
\frac{\partial^{2}{L(\boldsymbol{w})}}{\partial{\boldsymbol{w}}\partial{\boldsymbol{w'}}}=\boldsymbol{X'}(\boldsymbol{Z\bullet X})
\end{equation*}

\subsection{梯度下降}
参数更新的方法为
\begin{equation*}
\boldsymbol{w}+=\eta \frac{\partial{L(\boldsymbol{w})}}{\partial{\boldsymbol{w}}} -  \lambda \boldsymbol{w}
\end{equation*}

\subsection{牛顿法}
参数更新的方法为
\begin{equation*}
\boldsymbol{w}+=(\frac{\partial^{2}{L(\boldsymbol{w})}}{\partial{\boldsymbol{w}}\partial{\boldsymbol{w'}}})^{-1}
\frac{\partial{L{\boldsymbol{w}}}}{\partial{\boldsymbol{w}}} 
-\lambda \boldsymbol{w}
\end{equation*}

\subsection{数据结构}
使用ndarray作为主要的数据结构,运用pandas库处理UCI数据集。

\section{实验结果与分析}
\subsection{梯度下降}
\linespread{1.2}
\begin{table}[H]  
  \centering  
  \begin{threeparttable}  
  \caption{梯度下降分类准确率}  
  \label{tab:performance_comparison} 
  \begin{tabular}{m{0.17\textwidth}<{\centering} m{0.17\textwidth}<{\centering} m{0.17\textwidth}<{\centering} m{0.17\textwidth}<{\centering} m{0.17\textwidth}<{\centering}}
    \toprule[1.5pt]  
    \multirow{2}{*}{正则项系数}&\multicolumn{2}{c}{混合高斯分布} &\multirow{2}{*}{iris数据集} &\multirow{2}{*}{adult数据集}\cr  
    \cmidrule(lr){2-3}  
    &独立&不独立\cr  
    \midrule  
0         &0.97		&0.98		&1.0	&0.8069\cr
$e^{-10}$ &0.99		&1.0		&1.0	&0.8069\cr
$e^{-8}$  &0.96		&1.0		&1.0	&0.8080\cr
$e^{-6}$  &0.99		&0.97		&1.0	&0.8069\cr
$e^{-4}$  &0.99		&1.0		&1.0	&0.8080\cr
    \bottomrule[1.5pt]  
    \end{tabular}  
    \end{threeparttable}  
\end{table}

梯度下降此时分类效果较好,在adult数据集上准确率均在0.80左右,符合数据集附带文件上提到的错误率。


\subsection{牛顿法}
\linespread{1.2}
\begin{table}[H]  
  \centering  
  \begin{threeparttable}  
  \caption{牛顿法分类准确率}  
  \label{tab:performance_comparison} 
  \begin{tabular}{m{0.22\textwidth}<{\centering} m{0.22\textwidth}<{\centering} m{0.22\textwidth}<{\centering} m{0.22\textwidth}<{\centering}}
    \toprule[1.5pt]  
    \multirow{2}{*}{正则项系数}&\multicolumn{2}{c}{混合高斯分布} &\multirow{2}{*}{iris数据集}\cr  
    \cmidrule(lr){2-3}  
    &独立&不独立\cr  
    \midrule  
0         &0.99 	&0.93		&1.0	\cr
$e^{-6}$  &0.99   &0.99		&1.0	\cr
$e^{-4}$  &1.0    &0.96		&1.0	\cr
$e^{-3}$  &0.97		&0.96		&1.0	\cr
$e^{-2}$  &0.99		&0.98		&1.0	\cr
    \bottomrule[1.5pt]  
    \end{tabular}  
    \end{threeparttable}  
\end{table}


\linespread{1.2}
\begin{table}[h]
\centering
\caption{牛顿法分类adult数据集}
\label{tab:performance_comparison}
\begin{tabular}{ccccc}
\toprule[1.5pt]
\makebox[0.2\textwidth][c]{正则项系数}
&\makebox[0.16\textwidth][c]{0} 
&\makebox[0.16\textwidth][c]{$e^{-12}$}
&\makebox[0.16\textwidth][c]{$e^{-10}$}
&\makebox[0.16\textwidth][c]{$e^{-8}$}   \\\hline
正确率&0.7519&0.7519&0.7519&0.7519\\
\bottomrule[1.5pt]
\end{tabular}
\end{table}
牛顿法在处理adult数据集时,在求逆矩阵的过程中若正则项系数较大会在求逆过程中出现错误。


\section{结论}
混合高斯分布中,即使不满足朴素贝叶斯的条件,但只要各属性间的相关性适当,仍然可以保证模型的效果。
iris数据集,两个类别非常容易区分,符合数据集附带文件的说明。
adult数据集,数据维数多,训练用时长,数据噪声大,根据附带文件的说明,使用朴素贝叶斯的思想分类,误差大约为0.16,上述训练数据较为合理。

\begin{thebibliography}{1}
\bibitem{wiki}
CSDN,
https://blog.csdn.net/alanwalker1/article/details/112688367,
last accessed 2021/10/24
\end{thebibliography}

\newpage
\begin{appendix}
\section{主程序}
\begin{lstlisting}[language=python]
import copy

import numpy as np

from data_guass import get_data_guass
from data_uci import get_data_uci

gradient_regulation_radio = 0
newton_regulation_radio = 0


def sigmoid_trick(m):
    for i in range(len(m)):
        a = m[i]
        if a >= 0:
            m[i] = 1 / (1 + np.exp(-1 * a))
        else:
            temp = np.exp(a)
            m[i] = temp / (1 + temp)
    return m


def gradient_descent(train_X, train_Y, validation_X, validation_Y, regulation_radio=0.0):
    dimension_num = len(train_X[0])

    # adam中的超参数,一般使用以下值
    adam_beta_1 = 0.9
    adam_beta_2 = 0.99
    inf_small = 1e-8
    learning_rate = 0.01

    # 计算过程中的累计值
    m = np.zeros(dimension_num)
    v = np.zeros(dimension_num)
    accumulative_adam_beta_1 = adam_beta_1
    accumulative_adam_beta_2 = adam_beta_2

    w = np.array([0.1 for _ in range(dimension_num)])
    temp_w = copy.deepcopy(w)
    accuracy = test(validation_X, w, validation_Y)
    accuracy_list = [accuracy]
    iteration_times = 0

    while True:
        for i in range(10000):
            gradient = np.dot(train_X.T, sigmoid_trick(-1 * np.dot(train_X, w)) - train_Y) - regulation_radio * w

            m = adam_beta_1 * m + (1 - adam_beta_1) * gradient
            v = adam_beta_2 * v + (1 - adam_beta_2) * (gradient ** 2)
            M = m / (1 - accumulative_adam_beta_1)
            V = v / (1 - accumulative_adam_beta_2)
            accumulative_adam_beta_1 *= adam_beta_1
            accumulative_adam_beta_2 *= adam_beta_2
            w += learning_rate * M / (V ** (1 / 2) + inf_small)

        iteration_times += 1
        accuracy = test(validation_X, w, validation_Y)

        if accuracy < accuracy_list[-1] or iteration_times >= 15:
            break
        else:
            temp_w = copy.deepcopy(w)
            accuracy_list.append(accuracy)

    return temp_w, iteration_times


def Newton_method(train_X, train_Y, validation_X, validation_Y, regulation_radio=0.0):
    dimension_num = len(train_X[0])
    w = np.array([0.1 for _ in range(dimension_num)])
    temp_w = copy.deepcopy(w)
    accuracy = test(validation_X, w, validation_Y)
    accuracy_list = [accuracy]
    iteration_times = 0

    while True:
        for i in range(10000):
            sigmoid = sigmoid_trick(-1 * np.dot(train_X, w))
            a = np.dot(train_X.T, sigmoid - train_Y)
            b = train_X.T @ (np.reshape(sigmoid * (1 - sigmoid), (len(sigmoid), 1)) * train_X)
            delta_w = np.dot(np.linalg.pinv(b), a) - w * regulation_radio
            w += delta_w

        iteration_times += 1
        accuracy = test(validation_X, w, validation_Y)

        if iteration_times >= 15 or accuracy < accuracy_list[-1]:
            break
        else:
            temp_w = copy.deepcopy(w)
            accuracy_list.append(accuracy)

    return temp_w, iteration_times


def test(X, w, Y):
    X = np.array(X)
    Y = np.array(Y)

    predict_result = np.dot(X, w)
    right_num = 0
    for i in range(len(X)):
        if (predict_result[i] >= 0 and Y[i] == 0) or (predict_result[i] < 0 and Y[i] == 1):
            right_num += 1

    return float(right_num) / len(X)


if __name__ == '__main__':
    train_X_list, train_Y_list, validation_X_list, validation_Y_list, test_X_list, test_Y_list = get_data_guass(100, 20, 100, 0.7, False)
    # train_X_list, train_Y_list, validation_X_list, validation_Y_list, test_X_list, test_Y_list = get_data_uci("iris")
    # train_X_list, train_Y_list, validation_X_list, validation_Y_list, test_X_list, test_Y_list = get_data_uci("adult")

    train_X = np.array(train_X_list)
    train_Y = np.array(train_Y_list)
    validation_X = np.array(validation_X_list)
    validation_Y = np.array(validation_Y_list)
    test_X = np.array(test_X_list)
    test_Y = np.array(test_Y_list)

    w1, iteration_times_1 = gradient_descent(train_X, train_Y, validation_X, validation_Y, gradient_regulation_radio)
    print((test(test_X_list, w1, test_Y_list)), w1, iteration_times_1)

    w2, iteration_times_1 = Newton_method(train_X, train_Y, validation_X, validation_Y, newton_regulation_radio)
    print(test(test_X, w2, test_Y), w2, iteration_times_1)
\end{lstlisting}


\section{生成混合高斯分布}
\begin{lstlisting}[language=python]
import numpy as np


def get_data_guass(train_size, validation_size, test_size, positive_sample_radio, naive=True):
    if naive:
        cov12 = 0.0
    else:
        cov12 = 0.2

    mean_array_positive = [1, 1]
    cov_matrix_positive = [[0.4, cov12], [cov12, 0.3]]

    mean_array_negative = [-1, -1]
    cov_matrix_negative = [[0.4, cov12], [cov12, 0.3]]

    train_X_list, train_Y_list = get_guass_distribution(train_size, int(train_size * positive_sample_radio), mean_array_positive, cov_matrix_positive,
                                                        mean_array_negative, cov_matrix_negative)
    validation_X_list, validation_Y_list = get_guass_distribution(validation_size, int(validation_size * positive_sample_radio), mean_array_positive,
                                                                  cov_matrix_positive, mean_array_negative, cov_matrix_negative)
    test_X_list, test_Y_list = get_guass_distribution(test_size, int(test_size * positive_sample_radio), mean_array_positive, cov_matrix_positive,
                                                      mean_array_negative, cov_matrix_negative)
    return train_X_list, train_Y_list, validation_X_list, validation_Y_list, test_X_list, test_Y_list


def get_guass_distribution(total_size, positive_size, mean_array_positive, cov_matrix_positive, mean_array_negative, cov_matrix_negative):
    X_list = [(0.0, 0.0)] * total_size
    X_list[:positive_size] = np.random.multivariate_normal(mean_array_positive, cov_matrix_positive, size=positive_size)
    X_list[positive_size:] = np.random.multivariate_normal(mean_array_negative, cov_matrix_negative, size=total_size - positive_size)
    X_list = [list(sample) for sample in X_list]
    Y_list = [1] * positive_size + [0] * (total_size - positive_size)
    return X_list, Y_list
\end{lstlisting}

\section{读取uci数据集}
\begin{lstlisting}[language=python]
import copy

import numpy as np
import pandas as pd

from data.config.adult_config import adult_config
from data.config.iris_config import iris_config


def get_data_uci(uci_data_name):
    if uci_data_name == "iris":
        data_config = iris_config
        train_nrows = 100
        test_nrows = 100
        validation_index = [25, 76]
    elif uci_data_name == "adult":
        data_config = adult_config
        train_nrows = 1000
        test_nrows = 1000
        validation_index = [0, 500]
    else:
        print("尚无此数据集")
        return None

    train_filename = data_config[0]
    test_filename = data_config[1]
    field_list = data_config[2]
    discrete_field_dict = data_config[3]
    label_dict = data_config[4]
    get_log_list = data_config[5]

    train_X_list, train_Y_list = read_uci_data(train_filename, train_nrows, field_list, discrete_field_dict, label_dict, get_log_list)
    if test_filename == train_filename:
        test_X_list, test_Y_list = train_X_list, train_Y_list
    else:
        test_X_list, test_Y_list = read_uci_data(test_filename, test_nrows, field_list, discrete_field_dict, label_dict, get_log_list)
    validation_X_list = test_X_list[validation_index[0]:validation_index[1]]
    validation_Y_list = test_Y_list[validation_index[0]:validation_index[1]]

    return train_X_list, train_Y_list, validation_X_list, validation_Y_list, test_X_list, test_Y_list


def read_uci_data(filename, nrows, field_list, discrete_field_dict, label_dict, get_log_list):
    columns = copy.deepcopy(field_list)
    columns.append("label")
    raw_data = pd.read_csv(filename, names=columns, nrows=nrows)

    processed_data = []
    processed_data_label = []

    for i in range(len(raw_data)):
        processed_sample = []
        raw_sample = raw_data.iloc[i]

        try:
            for field in field_list:
                if field in discrete_field_dict:
                    processed_sample.append(discrete_field_dict[field][raw_sample.loc[field]])
                else:
                    processed_sample.append(float(raw_sample.loc[field]))
        except KeyError:
            continue
        except ValueError:
            continue

        processed_data_label.append(label_dict[raw_sample.loc["label"]])
        processed_data.append(processed_sample)

    if len(get_log_list) != 0:
        for sample in processed_data:
            for index in get_log_list:
                if sample[index] <= 10:
                    sample[index] = 1
                else:
                    sample[index] = np.log10(sample[index])

            sample.insert(0, 1)
    else:
        for sample in processed_data:
            sample.insert(0, 1)

    return processed_data, processed_data_label
\end{lstlisting}

\section{关于iris数据集的配置文件}
\begin{lstlisting}[language=python]
train_filename = "./data/iris/iris.data"
test_filename = "./data/iris/iris.data"
field_list = ["sepal_length", "sepal_width", "petal_length", "petal_width"]
discrete_field_dict = {}
label_dict = {"Iris-setosa": 0, "Iris-versicolor": 1}
get_log_list = []

iris_config = [train_filename, test_filename, field_list, discrete_field_dict, label_dict, get_log_list]
\end{lstlisting}

\section{关于adult数据集的配置文件}
\begin{lstlisting}[language=python]
work_class_dict = {' Private': 0, ' Self-emp-not-inc': 1, ' Self-emp-inc': 2, ' Federal-gov': 3, ' Local-gov': 4, ' State-gov': 5, ' Without-pay': 6,
                   ' Never-worked': 7, }
education_dict = {' Bachelors': 0, ' Some-college': 1, ' 11th': 2, ' HS-grad': 3, ' Prof-school': 4, ' Assoc-acdm': 5, ' Assoc-voc': 6, ' 9th': 7,
                  ' 7th-8th': 8, ' 12th': 9, ' Masters': 10, ' 1st-4th': 11, ' 10th': 12, ' Doctorate': 13, ' 5th-6th': 14, ' Preschool': 15, }
marital_status_dict = {' Married-civ-spouse': 0, ' Divorced': 1, ' Never-married': 2, ' Separated': 3, ' Widowed': 4, ' Married-spouse-absent': 5,
                       ' Married-AF-spouse': 6, }
occupation_dict = {' Tech-support': 0, ' Craft-repair': 1, ' Other-service': 2, ' Sales': 3, ' Exec-managerial': 4, ' Prof-specialty': 5,
                   ' Handlers-cleaners': 6, ' Machine-op-inspct': 7, ' Adm-clerical': 8, ' Farming-fishing': 9, ' Transport-moving': 10,
                   ' Priv-house-serv': 11, ' Protective-serv': 12, ' Armed-Forces': 13, }
relationship_dict = {' Wife': 0, ' Own-child': 1, ' Husband': 2, ' Not-in-family': 3, ' Other-relative': 4, ' Unmarried': 5, }
race_dict = {' White': 0, ' Asian-Pac-Islander': 1, ' Amer-Indian-Eskimo': 2, ' Other': 3, ' Black': 4, }
sex_dict = {' Female': 0, ' Male': 1, }
native_country_dict = {' United-States': 0, ' Cambodia': 1, ' England': 2, ' Puerto-Rico': 3, ' Canada': 4, ' Germany': 5,
                       ' Outlying-US(Guam-USVI-etc)': 6, ' India': 7, ' Japan': 8, ' Greece': 9, ' South': 10, ' China': 11, ' Cuba': 12, ' Iran': 13,
                       ' Honduras': 14, ' Philippines': 15, ' Italy': 16, ' Poland': 17, ' Jamaica': 18, ' Vietnam': 19, ' Mexico': 20,
                       ' Portugal': 21, ' Ireland': 22, ' France': 23, ' Dominican-Republic': 24, ' Laos': 25, ' Ecuador': 26, ' Taiwan': 27,
                       ' Haiti': 28, ' Columbia': 29, ' Hungary': 30, ' Guatemala': 31, ' Nicaragua': 32, ' Scotland': 33, ' Thailand': 34,
                       ' Yugoslavia': 35, ' El-Salvador': 36, ' Trinadad&Tobago': 37, ' Peru': 38, ' Hong': 39, ' Holand-Netherlands': 40, }

train_filename = "./data/adult/adult.data"
test_filename = "./data/adult/adult.test"
field_list = ['age', 'work_class', 'fnlwgt', 'education', 'education_num', 'marital_status', 'occupation', 'relationship', 'race', 'sex',
              'capital_gain', 'capital_loss', 'hours_per_week', 'native_country']

discrete_field_dict = {'work_class': work_class_dict, 'education': education_dict, 'marital_status': marital_status_dict,
                       'occupation': occupation_dict, 'relationship': relationship_dict, 'race': race_dict, 'sex': sex_dict,
                       'native_country': native_country_dict}
label_dict = {" <=50K": 0, " >50K": 1, " <=50K.": 0, " >50K.": 1}
get_log_list = [2, 10, 11]

adult_config = [train_filename, test_filename, field_list, discrete_field_dict, label_dict, get_log_list]
\end{lstlisting}

\end{appendix}
\end{document}
